\documentclass[../bericht.tex]{subfiles}

\begin{document}
  \chapter{Physikalische Grundlagen}

    \section{Zeeman Effekt}
    \label{sec:zeeman}

      Der \textsc{Zeeman} Effekt beschreibt die beobachtete Energieaufspaltung der entarteten $E_{n,l,m}$-Zustände der Elektronen in der Atomschale in $m$ Subniveaus in einem externen Magnetfeld $\vec{B}$, wobei $n,l,m$ die bekannten Quantenzahlen sind. Nach dem semiklassischen Modell bewegen sich die Elektronen auf einer Kreisbahn um den Atomkern. Hierdurch wird ein Kreisstrom und damit ein magnetisches Moment
      \begin{equation*}
        \vec{p}_m=-\frac{e}{2m_e}\cdot \vec{l},
      \end{equation*}
      mit dem Bahndrehimpuls des Elektrons $\vec{l}$, der Elementarladung $e$ und der Elektronenmasse $m_e$, erzeugt, welches mit dem Feld $\vec{B}$ wechselwirkt. Die magnetische Quantenzahl $m$ kann die Werte
      \begin{equation*}
        -l \le m \le +l
      \end{equation*}
      annehmen. Damit spalten die Energieniveaus $E_{n,l,m}$ gemä\ss
      \begin{equation}
        E_{n,l,m}=E_\mathrm{Coul}(n,l)+\underbrace{\frac{e\hslash}{2m_e}}_{=\mu_\mathrm{Bohr}}\cdot mB
      \end{equation}
      in $(2l+1)$ Komponenten auf. $\mu_\mathrm{Bohr}$ wird als \textsc{Bohr}'sches Magneton bezeichnet. Die \textit{\textsc{Zeeman}-Aufspaltung} der Spektrallinien ist unabhängig von $n,l$ und damit äquidistant
      \begin{equation}
        \Delta E=\mu_\mathrm{Bohr}\cdot B.
      \end{equation}
      Ausführlich inklusive der Erklärung der erlaubten Übergänge wird der \textsc{Zeeman} Effekt in \cite{dem:exp3-normaler-zeeman} beschrieben, obige Beschreibung enthält Auszüge.


    \section{Spin-Bahn-Kopplung und Feinstruktur}
    \label{sec:feinstruktur}

      Zusätzlich zu dem aus der Bahnbewegung des Elektrons resultierenden magnetischen Moment ergibt sich aus dem quantenmechanischen Modell das magnetische Moment
      \begin{equation*}
        \vec{\mu}_s=-2\mu_\mathrm{Bohr}\frac{\vec{s}}{\hslash},
      \end{equation*}
      welchem der Spindrehimpuls $\vec{s}$ zugrunde liegt.

      Das magnetische Moment $\vec{\mu}_s$ des rotierenden Elektrons befindet sich nun im durch die Rotation erzeugten Magnetfeld $\vec{B}$. Je nach Spineinstellung führt dies zur einer Erhöhung bzw. einer Verringerung der Energie
      \begin{equation}
        \Delta E_{l,s}=-\vec{\mu}_s\cdot \vec{B}\approx \frac{\mu_0Z\cdot e^2}{8\pi m_e^4r^3}\left( \vec{s}\cdot \vec{l}\right).
        \label{eq:feinstruktur-1}
      \end{equation}
      Hierbei ist $\mu_0$ die magnetische Suszeptibilität und $Z$ die Ordnungszahl des Atoms.

      Die vektorielle Addition von Bahndrehimpuls und Spindrehimpuls ergibt den Gesamtdrehimpuls
      \begin{equation*}
        \vec{j}=\vec{l}+\vec{s}.
      \end{equation*}
      Wie Bahndrehimpuls und Spindrehimpuls ist der Gesamtdrehimpuls gequantelt. Es gilt
      \begin{equation*}
        |\vec{j}|=\sqrt{j(j+1)}\cdot \hslash
      \end{equation*}
      wobei
      \begin{equation*}
        j=+\frac{1}{2} \quad \text{für} \quad l=0
      \end{equation*}
      und
      \begin{equation*}
        j=l\pm \frac{1}{2} \quad \text{für} \quad l>0,
      \end{equation*}
      da sich die $z$-Komponenten der Drehimpulse entweder parallel oder antiparallel einstellen können.

      Nun lässt sich \eqref{eq:feinstruktur-1} umschreiben zu
      \begin{equation}
        \Delta E_{l,j}=\frac{a}{2}\cdot \left[j(j+1)-l(l+1)-s(s+1)\right]
      \end{equation}
      mit der Spin-Bahn-Kopplungskonstante
      \begin{equation*}
        a=\frac{\mu_0 Z e^2\hslash ^2}{8\pi m_e^2r^3}.
      \end{equation*}
      Die Aufspaltung der Spektrallinien gemä\ss
      \begin{equation}
        E_{n,l,j}=E_n + \Delta E_{l,j}
      \end{equation}
      ist die sogenannte \textit{Feinstrukturaufspaltung}.

      Die vorangegangenen Ausführungen sind eine Kurzfassung von \cite{dem:exp3-feinstruktur}.


    \section{Hyperfeinstruktur}
    \label{sec:hyperfeinstruktur}

      Analog zum Spin des Elektrons wird auch dem räumlich ausgedehnten Atomkern ein Spin zugeordnet, der sogenannte Kernspin $\vec{I}$. Mit der Kernspinquantenzahl $I$ wird die Quantelung des Kernspins gemä\ss
      \begin{equation*}
        |\vec{I}|=\sqrt{I(I+1)}\hslash.
      \end{equation*}
      Dabei kann die Projektion auf die $z$-Richtung die $(2I+1)$ Werte
      \begin{equation*}
        I_z=m_I\cdot \hslash \quad \text{mit}\quad -I\le m_I \le +I
      \end{equation*}
      annehmen. Das magnetische Kernmoment ergibt sich damit zu
      \begin{equation*}
        \vec{\mu}_I=\gamma_\mathrm{K}\cdot \vec{I}.
      \end{equation*}
      Das magnetische Kernmoment befindet sich im vom Elektron durch Bahnbewegung und Spinmoment erzeugten Magnetfeld $B_j$ und hat hierdurch die Energie
      \begin{equation*}
        E_{I,j}=-|\mu_I|\cdot B_j \cdot \cos \left( \sphericalangle \left( \vec{j}, \vec{I}\right) \right).
      \end{equation*}
      Mit dem Gesamtdrehimpuls des Atoms
      \begin{equation*}
        \vec{F} = \vec{j}+\vec{I}
      \end{equation*}
      spaltet sich damit jedes Energieniveau der Feinstruktur in der \textit{Hyperfeinstruktur} gemä\ss
      \begin{equation}
        E_{HFS}=E{n,l,j} + \frac{A}{2}\left[ F(F+1) - j(j+1) - I(I+1) \right],
      \end{equation}
      mit der Hyperfeinkonstante
      \begin{equation*}
        A=\frac{g_I \cdot \mu_\mathrm{K}\cdot B_j}{\sqrt{j(j+1)}}.
      \end{equation*}
      Die vollständige Herleitung ist in \cite{dem:exp3-hyperfeinstruktur} zu finden.



\end{document}
