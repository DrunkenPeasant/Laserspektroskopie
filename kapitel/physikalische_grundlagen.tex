\documentclass[../bericht.tex]{subfiles}

\begin{document}
  \chapter{Physikalische Grundlagen}

    \section{Zeemann Effekt}
    \label{sec:zeemann}


    \section{Spin-Bahn-Kopplung und Feinstruktur}
    \label{sec:feinstruktur}

      Im semiklassischen Modell bewegen sich die Elektronen auf diskreten Kreisbahnen mit Radius $r$ um den Atomkern. Als Ladungsträger mit Ladung $q_e=-e$ stellt das rotierende Elektron einen Kreisstrom dar. Dieser erzeugt ein magnetisches Dipolmoment
      \begin{equation*}
        \vec{\mu}_l=-\underbrace{\frac{e\hslash}{2m_e}}_{=\mu_\mathrm{Bohr}}\frac{\vec{l}}{\hslash}
      \end{equation*}
      mit dem \textsc{Bohr}'schen Magneton, welches sich neben der Elektronenladung aus der Elektronenmasse $m_e$ und dem reduzierten \textsc{Planck}'schen Wirkungsquantum zusammensetzt. Weiter ist $\vec{l}$ der Bahndrehimpuls des Elektrons.

      Aus dem quantenmechanischen Modell ergibt sich außerdem das magnetische Moment
      \begin{equation*}
        \vec{\mu}_s=-2\mu_\mathrm{Bohr}\frac{\vec{s}}{\hslash},
      \end{equation*}
      welchem der Spindrehimpuls $\vec{s}$ zugrunde liegt.

      Das magnetische Moment $\vec{\mu}_s$ des rotierenden Elektrons befindet sich nun im durch die Rotation erzeugten Magnetfeld $\vec{B}$. Je nach Spinstellung führt dies zur einer Erhöhung bzw. einer Verringerung der Energie
      \begin{equation}
        \Delta E_{l,s}=-\vec{\mu}_s\cdot \vec{B}\approx \frac{\mu_0Z\cdot e^2}{8\pi m_e^4r^3}\left( \vec{s}\cdot \vec{l}).
        \label{eq:feinstruktur-1}
      \end{equation}
      Hierbei ist $\mu_0$ die magnetische Suszeptibilität und $Z$ die Ordnungszahl des Atoms.

      Die vektorielle Addition von Bahndrehimpuls und Spindrehimpuls ergibt den Gesamtdrehimpuls
      \begin{equation*}
        \vec{j}=\vec{l}+\vec{s}.
      \end{equation*}
      Wie Bahndrehimpuls und Spindrehimpuls ist der Gesamtdrehimpuls gequantelt. Es gilt
      \begin{equation*}
        |\vec{j}|=\sqrt{j(j+1)}\cdot \hslash
      \end{equation*}
      wobei
      \begin{equation*}
        j=+\frac{1}{2} \quad \text{für} \quad l=0
      \end{equation*}
      und
      \begin{equation*}
        j=l\pm \frac{1}{2} \quad \text{für} \quad l>0,
      \end{equation*}
      da sich die $z$-Komponenten der Drehimpulse entweder parallel oder antiparallel einstellen können.

      Nun lässt sich \eqref{eq:feinstruktur-1} umschreiben zu
      \begin{equation}
        \Delta E_{l,j}=\frac{a}{2}\cdot \left[j(j+1)-l(l+1)-s(s+1)\right]
      \end{equation}
      mit der Spin-Bahn-Kopplungskonstante
      \begin{equation*}
        a=\frac{\mu_0 Z e^2\hslash ^2}{8\pi m_e^2r^3}.
      \end{equation*}
      Die Aufspaltung der Spektrallinien gemäß
      \begin{equation}
        E_{n,l,j}=E_n + \Delta E_{l,j}
      \end{equation}
      ist die sogenannte Feinstrukturaufspaltung.


    \section{Hyperfeinstruktur}
    \label{sec:hyperfeinstruktur}


\end{document}
